\documentclass[a4paper,12pt]{article}

\usepackage{graphicx}

\usepackage{epsfig}

\usepackage{amsmath}

\usepackage{amssymb}

\usepackage{amsthm}

\usepackage{bbm}

% --------------

\title{Detailed method to solve the charge state equations}

\author{Nicolas Winckler}

\begin{document}

\maketitle

%\tableofcontents

\section{Introduction}

The rate of fraction of ion in the state $i$, $F_i$, is given by :
\begin{equation}
\frac{d}{dx}F_i = \sum_j \sigma_{ji} F_j - \sum_j \sigma_{ij} F_i
\label{charge-state-eq-gen}
\end{equation}

For an $N$-level system, equation (\ref{charge-state-eq-gen}) becomes :

\begin{equation}
\left\lbrace
\begin{array}{lcl}
\frac{dF_i}{dx}&=& \sum_{j=1}^{i-1} \sigma_{ji}^{EL} F_j  + \sum_{s=i+1}^N  \sigma_{si}^{EC} F_s   \\
 & & \\
 & - &   \left( \sum_{m=i+1}^{N} \sigma_{im}^{EL} +\sum_{k=1}^{i-1} \sigma_{ik}^{EC} \right) F_i \text{ , where } 1\leq i \leq N-1\\
 & & \\
1 &=&\sum_{i=1}^N F_i\\
\end{array}\right.
\end{equation}

These equations can be re-arranged in a linear system of equation of the following form :
\\ 

$$(S_0)\left\lbrace
\begin{array}{lcl}
F'_1&=& \sum_{i=1}^N m_{1,i} \ \ F_i  \\
 & & \\
... &=& ...\\
 & & \\
F'_{N-1} &=& \sum_{i=1}^N m_{N-1,i} \ \ F_i \\
 & & \\
1 &=&\sum_{i=1}^N F_i\\
\end{array}\right.$$
\\
where $m_{ij}$ are real and constant linear combination of the $\sigma_{pq}$, and where index $i$ has range from 1 to $N-1$, and index $j$ from 1 to $N$. 

We will consider two cases : - the equilibrium case, that is, $dF/dx=0$, and the general one, i.e. $dF/dx \neq 0$.
\section{Equlibrium case} 
\subsection{Equations}
At equilibrium  $dF/dx=0$ and the system $S_0$ reduces to:
$$(S'_0)\left\lbrace
\begin{array}{lcl}
0 &=& \sum_{i=1}^N m_{1,i} \ \ F_i  \\
 & & \\
... &=& ...\\
 & & \\
0 &=& \sum_{i=1}^N m_{N-1,i} \ \ F_i \\
 & & \\
1 &=&\sum_{i=1}^N F_i\\
\end{array}\right.$$

\subsection{Solution}
The solution of $S'_0$ is trivial :
\begin{equation}
\begin{pmatrix}
0 \\ 
 \vdots\\
1\\ \end{pmatrix} = A F \Leftrightarrow F=A^{-1} \begin{pmatrix}
0 \\ 
 \vdots\\
1\\ \end{pmatrix} = N^{th} \text{column of } A^{-1}
\end{equation}

\section{ General case} 
 \subsection{Equations}
 With $F_N=1-\sum_{i=1}^{N-1} F_i$, the system $S_0$ reduces, by substitution of $F_N$, to :

$$(S_1)\left\lbrace
\begin{array}{lcl}
F'_1&=& \sum_{i=1}^{N-1} (m_{1,i} -   m_{1,N} ) \ \ F_i  +  m_{1,N}\\
 & & \\
... &=& ...\\
 & & \\
F'_{N-1} &=& \sum_{i=1}^{N-1} (m_{N-1,i}-m_{N-1,N}) \ \ F_i   +  m_{N-1,N}\\
\end{array}\right.$$
\\
which can be rewritten as follow:
\begin{equation}
 F'_{k} = \sum_{i=1}^{N-1} a_{k,i} \ \ F_i   +  g_k
 \label{eqM1}
\end{equation}
where 
$$g_k = m_{k,N} = \sigma_{Nk}^{EC}  \text{ , with } 1<k<N$$
In matrix form it becomes:

%$$\frac{d}{dt}F=AF$$

\begin{equation}
\frac{d}{dx}F=AF + g, \\
\label{eqM2}
\end{equation}

with, \\
$\\ A=\begin{pmatrix}
a_{11} & a_{12} & \ldots & a_{1,N-1} \\
\vdots & \ddots & \ddots & \vdots  \\
\vdots & \ddots & \ddots & \vdots  \\
a_{N-1,1} & \ldots & \ldots & a_{N-1,N-1}
\end{pmatrix}$,  $F=\begin{pmatrix}
 F_1 \\ 
 F_2 \\
 \vdots\\
 F_{N-1} \\ \end{pmatrix}$,  and $g=\begin{pmatrix}
 g_1 \\ 
 g_2 \\
 \vdots\\
 g_{N-1} \\ \end{pmatrix}\\$,\\
 
 Note that this equation could be used to solve the equilibrium case (dF/dx=0) as well. In that case the solution is $F= - A^{-1} g$, with the dimensions of $F$, $A$, and $g$ equal to $N-1$.
 \subsection{Solutions}
 The equations  (\ref{eqM1}) and (\ref{eqM2}) are linear systems of differential equations with second (constant) members $g$. According to the Cauchy theorem, given the initial condition $F(x=0)=F_0$, there exists a unique solution to equation (\ref{eqM2}). The form of the solution is given by :
 
 
 \begin{equation}
F(x) = R_A(x)F_0 + Z(x)
\end{equation}
where $R_A(x)F_0$ is solution of the homogeneous equation : 
 \begin{equation}
\frac{d}{dx}F=AF , \\
\label{eqhomogen}
\end{equation}
 and $Z(x)$ is the particular solution of equation (\ref{eqM2}).
 \subsubsection{Homgeneous solution}

 \subsubsection*{Case 1 : $A$ diagonalizable in $\mathbb{R}$}
 If $A$ can be diagonalized then 
  \begin{equation}
 A=P D P^{-1},
 \end{equation}
 where $D$ is  diagonal (with eigenvalues $\lambda_i$) and $P$ is the eigenvector matrix. 
 With the change of variable: 
 \begin{equation}
 Y=P^{-1}F
 \end{equation}
 the equation  (\ref{eqhomogen}) becomes:
  \begin{equation}
\frac{d}{dx}Y=DY
 \end{equation}
The solution to this equation have the form :\\

  \begin{equation}
  Y(x)=\begin{pmatrix}
 C_1 e^{\lambda_1 x} \\
  \vdots\\
 C_{N-1} e^{\lambda_{N-1} x}\end{pmatrix} \\
  \end{equation}
  
 where the $C_i$ are determined by the initial conditions : 
  \begin{equation}
 Y(x=0) =  \begin{pmatrix}
 C_1 \\
  \vdots\\
 C_{N-1}\end{pmatrix}  = P^{-1} F_0. 
 \end{equation}
 The solution is then obtained by 
   \begin{equation}
  F(x)=PY(x)
  \end{equation}
which is a linear combination of exponential functions.

 \subsubsection*{Case 2 : $A$ diagonalizable only in $\mathbb{C}$}
In that case the non-real eigen values are pairs of complex conjugate, and it is sufficient to replace in the generatrice familly 
$$ (\alpha e^{\lambda x} U ,  \beta e^{ \bar{\lambda} x} \bar{U}) \ \text{ with } \ (a\Re( e^{\lambda x} U ),  b \Im( e^{\lambda x} U) )$$
 where $U$ and $\bar{U}$ are pairs of complex conjugate eigenvectors of $P$ . 
\subsubsection*{Example:} 
If we only have one pair of complex eigenvalues, e.g. $\lambda_1 = \theta + \omega i $ and $\lambda_2 = \theta - \omega i $, then the matrix $P= \{ v_1,v_2, ..., v_{N-1}\}$ will have one pair of complex conjugate eigenvectors $v_1$ and $v_2$ and the solution will be
 
 \begin{equation}
 F = (C_1 \Re( e^{i \omega  x}  \times v_1) + C_2 \Im(e^{i \omega x} \times v_1 ) e^{\theta x} +  \sum_{i \neq 1,2} C_i e^{\lambda_i x} v_i
\end{equation}  
Let's write the complex elements of the vector $v_1$ as $a_k + i b_k$ . Then, the complex eigen vectors contribution can be written as follows :
 \begin{equation}
 C_1  e^{\theta x} \begin{pmatrix}
 a_1 \cos\omega x - b_1 \sin \omega x \\ 
 a_2 \cos\omega x - b_2 \sin \omega x\\
 \vdots\\
 a_{N-1} \cos\omega x - b_{N-1} \sin\omega x \\ \end{pmatrix} + 
 C_2  e^{\theta x} \begin{pmatrix}
 a_1 \sin \omega x + b_1 \cos \omega x \\ 
 a_2 \sin \omega x + b_2 \cos \omega x\\
 \vdots\\
 a_{N-1} \sin \omega x + b_{N-1} \cos \omega x \\ \end{pmatrix} 
 \end{equation}
 The unknown coefficients $C_i$ can be determined by the initial condition 
 $$ Y(x=0) = P_{\Re}^{-1} F_0 $$
 where $P_{\Re} = \{ \Re{(v_1)}, \Im{(v_1)}, v_3, ... , v_{N-1} \}$.


 
 \subsubsection*{Case 3 : $A$ only triangularizable}
 In case $A$ is not diagonalizable but only triangularizable, then equation ($\ref{eqhomogen}$) can be written as:
 
  \begin{equation}
\frac{d}{dx}F=P T P^{-1} F , \\
\label{eqhomogenTriangle}
\end{equation}

Then we can solve the last equation $Y'_{N-1} = \lambda_{N-1} Y_{N-1}$, and report the solution in equation $N-2$, and so on.
 
\subsubsection{Particular solution}
In case the particular solution of the system is unknown, we need to search for one. The system in equation (\ref{eqM2}) can be written as follow:

\begin{equation}
P^{-1} Z' = D P^{-1} Z + P^{-1} g  \Leftrightarrow \widetilde{Y'} = D \widetilde{Y} +  \widetilde{g}
\end{equation}
where $Z$  and $\widetilde{Y}= P^{-1} Z$ are the particular solutions of the system and 
\begin{equation}
\widetilde{g} = \begin{pmatrix}
\widetilde{g_1} \\
  \vdots\\
 \widetilde{g_{N-1}}\end{pmatrix} = P^{-1} g
\end{equation}

Then we can find the particular solution by integrating each line:

$$(S_p)\left\lbrace
\begin{array}{lcl}
\widetilde{Y'_1} &=& \lambda_1 \widetilde{Y_1} + \widetilde{g_1} \\
 & & \\
... &=& ...\\
 & & \\
\widetilde{Y'_{N-1}} &=& \lambda_{N-1} \widetilde{Y_{N-1}} + \widetilde{g_{N-1}}\\
\end{array}\right.$$
since $\widetilde{g}$ is constant the particular solution is given by:

\begin{equation}
\widetilde{Y_k} = \left\lbrace
\begin{array}{lcl}
 \widetilde{g_k}/\lambda_k 		& \text{if}		& \lambda_k \neq 0\\
  \widetilde{g_k} x 						& \text{if}		& \lambda_k = 0\\
\end{array}\right.
\end{equation}
and the particular solution of the system (\ref{eqM2}) is given by:
\begin{equation}
Z(x) = P \widetilde{Y}
\end{equation}

\subsection{Summary}
In order to solve the system (\ref{eqM2}) we proceed as follow:
\begin{itemize}
\item Solve the homogeneous system given in the equation (\ref{eqhomogen}) :
	\begin{itemize}
	\item Search for the eigenvalues and eigen vectors, and : 
			\begin{itemize}
			\item case 1: Diagonalize $A$ in $\mathbb{R}$
			\item case 2: Diagonalize $A$ in $\mathbb{C}$ 
			\item case 3: Triangularize $A$ in $\mathbb{C}$
			\end{itemize}
	\item Change the variable $Y=P^{-1}F$
	\item Given the initial condition $F(x=0)=F_0$ , the solution of the homogeneous equation is $R_A(x)F_0$ and is a linear combination of exponential functions (or $\cos$, $\sin$ if complex eigenvalues).
	\end{itemize}
\item Search for particular solution $Z(x)$, which is either a 1-order or 0-order polynomial function, depending on the eigenvalue.
\item The final solution is given by $F(x) = R_A(x)F_0 + Z(x)$
\end{itemize}


\end{document}